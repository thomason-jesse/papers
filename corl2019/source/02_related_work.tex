
Dialogs in \dataset{} begin with an underspecified, ambiguous instruction analogous to what robots may encounter in a home environment (e.g., ``Go to the room with the bed'').
Dialogs include both navigation and question asking / answering to guide the search, akin to a robot agent asking for clarification when moving through a new environment.
Table~\ref{tab:rw_comparison} summarizes how \dataset{} combines the strengths and difficulties of a subset of existing navigation and question answering tasks.

\paragraph{Vision-and-Language Navigation.}
% MARCO http://www.cs.utexas.edu/users/ml/clamp/navigation/
% DRIF https://github.com/lil-lab/drif
% R2R https://arxiv.org/pdf/1711.07280.pdf
% Touchdown https://arxiv.org/pdf/1811.12354.pdf
Early, simulator-based Vision-and-Language Navigation (VLN) tasks use language instructions that are unambiguous---designed to uniquely describe the goal---and fully specified---describing the steps necessary to reach the goal~\cite{macmahon:aaai06,chen:aaai11}.
In a more recent setting, a simulated quadcopter drone uses low-level controls to follow a route described in natural language~\cite{blukis:corl18}.
In photorealistic simulation environments, agents can navigate high-definition scans of indoor scenes~\cite{anderson:cvpr18} or large, outdoor city spaces~\cite{chen:cvpr19}.
% EQA https://embodiedqa.org/paper.pdf
% IQA https://arxiv.org/abs/1712.03316
In interactive question answering~\cite{das:cvpr18,gordon2018iqa} settings, the language context is a single question (e.g., ``What color is the car?'') that requires navigation to answer.
The questions serve as underspecified instructions, but are unambiguous (e.g., there is only one car whose color can be asked about).
These questions are generated from templates rather than human language.
% How we differ.
In \dataset{}, input is an underspecified hint about the goal location (e.g., ``The goal room has a sink'') requiring exploration and dialog to resolve.
Rather than single instructions, \dataset{} uses two-sided, human-human dialogs.

\paragraph{Question Answering and Dialog.}
% CLEVR https://arxiv.org/pdf/1612.06890.pdf
% VQA https://arxiv.org/pdf/1505.00468.pdf hudson:cvpr18
% CLEVR-Dialog https://arxiv.org/pdf/1903.03166.pdf
% VisDial https://arxiv.org/pdf/1611.08669.pdf
In Visual Question Answering (VQA), agents answer language questions about a static image.
These tasks exist for templated language on rendered images~\cite{johnson:cvpr17} and human language on real-world images~\cite{antol:iccv15,hudson:cvpr18,zellers:cvpr19}.
Later extensions feature two-sided dialog, where a series of question-answer pairs provide context for the next question~\cite{kottur:naacl19,das:cvpr17}.
% QuAC https://arxiv.org/pdf/1808.07036.pdf
% CoQa https://arxiv.org/pdf/1808.07042.pdf
% Sharc https://arxiv.org/pdf/1809.01494.pdf
Question answering in natural language processing is a long-studied task for questions about static text documents (e.g., the Stanford QA Dataset~\cite{rajpurkar:emnlp16}).
Recently, this paradigm was extended to two-sided dialogs via human-human, question-answer pairs about a document~\cite{choi:emnlp18,saeidi:emnlp18,reddy:tacl19}.
% How we differ.
Questions in these datasets are unambiguous: they have a right answer that can be inferred from the context.
By contrast, \dataset{} conversations begin with a hint about the goal location that is always ambiguous and requires cooperation between participants.
Contrasting VQA, because \dataset{} extends navigation the visual context is temporally dynamic---new visual observations arrive at each timestep.

\begin{table}[ht!]
\centering
\begin{small}
\begin{tabular}{lccccccc}
    \textbf{Dataset} & \multicolumn{4}{c}{\textbf{---Language Context---}} & \multicolumn{3}{c}{\textbf{---Visual Context---}} \\
    & Human & Amb & UnderS & Temporal & Real-world & Temporal & Shared \\
    \toprule
MARCO\cite{macmahon:aaai06,chen:aaai11}, DRIF\cite{blukis:corl18} & \cmark & \xmark & \xmark & \bad{1I} & \xmark & \good{Dynamic} & - \\
    R2R\cite{anderson:cvpr18}, Touchdown\cite{chen:cvpr19} & \cmark & \xmark & \xmark & \bad{1I} & \cmark & \good{Dynamic} & - \\
    EQA\cite{das:cvpr18}, IQA\cite{gordon2018iqa} & \xmark & \xmark & \cmark & \bad{1Q} & \xmark & \good{Dynamic} & - \\
    CLEVR\cite{johnson:cvpr17} & \xmark & \xmark & - & \bad{1Q} & \xmark & \bad{Static} & - \\
    VQA\cite{antol:iccv15,hudson:cvpr18,zellers:cvpr19} & \cmark & \xmark & - & \bad{1Q} & \cmark & \bad{Static} & - \\
    CLEVR-Dialog\cite{kottur:naacl19} & \xmark & \xmark & - & \good{2D} & \xmark & \bad{Static} & \cmark \\
    VisDial\cite{das:cvpr17} & \cmark & \xmark & - & \good{2D} & \cmark & \bad{Static} & \cmark \\
    VLNA\cite{nguyen:cvpr19}, HANNA\cite{nguyen:emnlp19} & \xmark & \cmark & \cmark & \neutral{1D} & \cmark & \good{Dynamic} & \xmark \\
    TtW\cite{devries:arxiv18} & \cmark & \xmark & \cmark & \good{2D} & \cmark & \good{Dynamic} & \xmark \\
    \midrule
    \dataset{} & \cmark & \cmark & \cmark & \good{2D} & \cmark & \good{Dynamic} & \cmark \\
    \bottomrule
\end{tabular}
\end{small}
\caption{
Compared to existing datasets involving vision and language input for navigation and question answering, \dataset{} is the first to include two-sided dialogs held in natural language, with the initial navigation instruction being both ambiguous (\textit{Amb}) and underspecified (\textit{UnderS}), and situated in a photorealistic, visual navigation environment viewed by both speakers.
For temporal language context, we note single navigation instructions (\textit{1I}) and questions (\textit{1Q}) versus 1-sided (\textit{1D}) and 2-sided (\textit{2D}) dialogs.
}
\vspace{-6mm}
\label{tab:rw_comparison}
\end{table}

\paragraph{Task-oriented Dialog.}
% Joyce object identification https://www.aaai.org/ocs/index.php/AAAI/AAAI15/paper/view/10000/9555 liu:aaai15
% Tellex inverse semantics http://cs.brown.edu/courses/csci2951-k/papers/tellex14.pdf tellex:rss14
% VLNA https://arxiv.org/pdf/1812.04155.pdf
In human-robot collaboration, robot language requests for human help can be generated to elicit non-verbal human help (e.g, moving a table leg to be within reach for the robot)~\cite{tellex:rss14}.
However, humans may use language to respond to robot requests for help in task-oriented dialogs~\cite{thomason:icra19,williams:auro19,marge:naacl19}.
Recent work adds requesting navigation help as an action, but the response either comes in the form of templated language that encodes gold-standard planner action sequences~\cite{nguyen:cvpr19} or as an automatic generation trained from human instructions and coupled with a visual goal frame as additional supervision~\cite{nguyen:emnlp19}.
Past work introduced Talk the Walk (TtW)~\cite{devries:arxiv18}, where two humans communicate to reach a goal location in a photorealistic, outdoor environment.
In TtW, the guiding human does not have an egocentric view of the environment, but an abstracted semantic map, and so language grounding centers around semantic elements like ``bank'' and ``restaurant'' rather than visual features, and the target location is unambiguously shown to the guide from the start.
In \dataset{}, a \nav{} human generates language requests for help, and an \ora{} human answers in language conditioned on higher-level, visual observations of what a shortest-path planner would do next, with both players observing the same, egocentric visual context.
In some ways, \dataset{} echoes several older human-human, spoken dialog corpora like the HCRC Map Task~\cite{anderson:ls91}, SCARE~\cite{stoia:lrec08}, and CReST~\cite{eberhard:lrec10}, but these are substantially smaller and have fewer and less rich environments.

\paragraph{Background: Matterport Simulator and the Room-2-Room Task.}
We build on the R2R task~\cite{anderson:cvpr18} and train navigation agents using the same simulator and API.
MatterPort contains 90 3D house scans, with each scan $S$ divided into visual panoramas $p\in S$ (nodes which a navigation agent can occupy) accompanied by an adjacency matrix $A_S$.
We differentiate between the \textit{steps} and \textit{distance} between $p$ and $q$---\textit{steps} represent the number of intervening nodes $d_h$, while \textit{distance} is defined in meters as $d_m$.
Step distance $d_h(p, q)$ is the number of hops through $A_S$ to get from node $p$ to node $q$.
The distance in meters $d_m(p, q)$ is defined as physical distance if $A_S[p,q] = 1$ or the shortest route between $p$ and $q$ otherwise.
On average, $1$ step corresponds to $2.25$ meters.

At each timestep, an agent emits a navigation action taken in the simulated environment.
The actions are to turn \textit{left} or \textit{right}, tilt \textit{up} or \textit{down}, move \textit{forward} to an adjacent node, or \textit{stop}.
After taking any action except \textit{stop}, the agent receives a new visual observation from the environment.
The \textit{forward} action is only available if the agent is facing an adjacent node.
